\appendix
%Appendix A
\section{Examples}
In this section, we provide two examples to illustrate the usage of Blaze.
All the source code of our implementation is included in our GitHub repository~\cite{blaze}.

\subsection{Word frequency count}
\label{app:wordcount}
In this example, we count the number of occurrences of each unique word in an input file with Blaze MapReduce.
We save the results in a distributed hash map, which can be used for further processing.

To compile this example, you can clone our repository~\cite{blaze}, go to the \lstinline{example} folder and type \lstinline{make wordcount}.

\begin{lstlisting}
#include <blaze/blaze.h>
#include <iostream>

int main(int argc, char** argv) {
  blaze::util::init(argc, argv);
  
  // Load file into distributed container.
  auto lines =
      blaze::util::load_file("filepath...");

  // Define mapper function.
  const auto& mapper = [&](
      const size_t,  // Line id.
      const std::string& line,
      const auto& emit) {
    // Split line into words.
    std::stringstream ss(line);
    std::string word;
    while (getline(ss, word, ' ')) {
      emit(word, 1);
    }
  };

  // Define target hash map.
  blaze::DistHashMap<std::string, size_t> words;

  // Perform mapreduce.
  blaze::mapreduce<
      std::string, std::string, size_t>(
          lines, mapper, "sum", words);
    
  // Output number of unique words.
  std::cout << words.size() << std::endl;
  
  return 0;
}
\end{lstlisting}

\subsection{Monte Carlo Pi Estimation}
\label{app:pi}

In this example, we present a MapReduce implementation of the Monte Carlo $\pi$ estimation.

To compile this example, you can clone our repository~\cite{blaze}, go to the \lstinline{example} folder and type \lstinline{make pi}.

\begin{lstlisting}
#include <blaze/blaze.h>
#include <iostream>

int main(int argc, char** argv) {
  blaze::util::init(argc, argv);
  
  const size_t N_SAMPLES = 1000000;

  // Define source.
  blaze::DistRange<size_t> samples(0, N_SAMPLES);

  // Define mapper.
  const auto& mapper = 
      [&](const size_t, const auto& emit) {
    // Random function in std is not thread safe.
    double x = blaze::random::uniform();
    double y = blaze::random::uniform();
    // Map points within circle to key 0.
    if (x * x + y * y < 1) emit(0, 1);
  };

  // Define target.
  std::vector<size_t> count(1);  // {0}

  // Perform MapReduce.
  blaze::mapreduce<size_t, size_t>(
      samples, mapper, "sum", count);

  std::cout << 4.0 * count[0] / N_SAMPLES
      << std::endl;
  
  return 0;
}
\end{lstlisting}


In conventional MapReduce implementations, mapping big data onto a single key is usually slow and consumes a large amount of memory during the map phase.
Hence, in practice, people usually hand-code parallel for loops in such situations.
However, by using Blaze, the above code has similar memory consumption and performance as the hand-optimized parallel for loops.
In short, Blaze frees users from dealing with low-level data communications while ensuring high performance.
